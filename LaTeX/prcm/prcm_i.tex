\documentclass[6pt]{article}
\usepackage[top=0.75in, bottom=0.75in, left=0.25in, right=0.25in]{geometry}
\usepackage{graphicx}
\usepackage{amsmath}
\usepackage[document]{ragged2e}

\usepackage[yyyymmdd]{datetime}
\renewcommand{\dateseparator}{-}

\usepackage{fancyhdr}
\pagestyle{fancy}
\fancyhf{}
\lhead{\Large{\textbf{Measuring \& Managing Equity Portfolio Risk}}}
\rhead{\texttt{Compiled on {\today} at \currenttime}}
\lfoot{\textit{Prepared for} \textbf{PRCM} \textit{by David Korsnack}}
\rfoot{Page \thepage}
 
\begin{document}

\textbf{\underline{Summary}} \\

\justify Systematic investment strategies that rely on measuring markets can
be improved by increasing the frequency of observations and/or the dimensionality
of the data set. Specifically, below are some examples of improving the risk-adjusted
return of the S\&P500 by reaching more stastically significant conclusions about the
index's risk profile: \\

\begin{center}
Risk Managed S\&P500 Strategies \\
\begin{tabular}{|c|ccccc|}
  \hline
  & Ann Ret \% & Ann Vol \% & Ann Ret/Vol & Max DD \% & Turnover \% \\
  \hline
  Buy-and-Hold SPY & 7.83 & 19.31 & 0.41 & 55.19 & 0.00 \\ 
  \^{}VIX Managed SPY, Intraday Frequency & 7.62 & 15.87 & 0.48 & 45.29 & 7.74 \\
  $\mathbf{\Sigma}$ Managed SPY, Intraday Frequency & 11.52 & 16.56 & 0.70 & 42.03 & 10.74 \\
  \hline
\end{tabular} \\
\end{center}

\vspace{1in}

\textbf{\underline{Current Measurements}} (as of 2017-02-08) \\

\begin{center}
Approximate Index Weights (\%) \\
\begin{tabular}{|ccccccccc|}
  \hline
  XLB & XLE & XLF & XLI & XLK & XLP & XLU & XLV & XLY \\
  \hline
  5.30 & 6.78 & 14.19 & 14.83 & 15.04 & 7.20 & 5.93 & 12.92 & 17.80 \\
  \hline
\end{tabular} 
\vspace{0.25in}

Intraday Frequency, 100 data point window \\
\begin{tabular}{|c|ccccccccc||c|}
  \hline
  $\rho_{xy}$ & XLB & XLE & XLF & XLI & XLK & XLP & XLU & XLV & XLY & SPY \\
  \hline
    XLB & 1.00 & 0.70 & 0.81 & 0.90 & 0.78 & 0.78 & 0.27 & 0.76 & 0.88 & 0.90 \\
    XLE & & 1.00 & 0.68 & 0.72 & 0.70 & 0.61 & 0.15 & 0.64 & 0.75 & 0.77 \\
    XLF & & & 1.00 & 0.88 & 0.75 & 0.79 & 0.24 & 0.75 & 0.84 & 0.88 \\
    XLI & & & & 1.00 & 0.79 & 0.81 & 0.31 & 0.84 & 0.87 & 0.92 \\
    XLK & & & & & 1.00 & 0.74 & 0.24 & 0.79 & 0.81 & 0.91 \\
    XLP & & & & & & 1.00 & 0.30 & 0.79 & 0.77 & 0.86 \\
    XLU & & & & & & & 1.00 & 0.23 & 0.21 & 0.28 \\
    XLV & & & & & & & & 1.00 & 0.79 & 0.89 \\
    XLY & & & & & & & & & 1.00 & 0.92 \\
  \hline
  \hline
  $\sigma_{x}$ (\%) & 19.41 & 20.15 & 21.80 & 18.96 & 20.85 & 14.22 & 12.27 & 18.69 & 18.86 & 16.99 \\
  \hline
\end{tabular}

\newpage

\Large{$\sigma=$ CBOE Volatility Index}
\includegraphics{SPY.png}
\includegraphics{^VIX.png}
\includegraphics{rv.png}
\includegraphics{acorr.png}

\newpage

\Large{$\sigma=$ CBOE Volatility Index}
\includegraphics{xx.png}
\includegraphics{cr.png}
\includegraphics{dd.png}
\includegraphics{v.png}

\newpage

\Large{$\sigma\overset{?}{=}$\ CBOE Volatility Index\ $\overset{?}{=}$\ Realized Volatility}
\includegraphics{SPY.png}
\includegraphics{^VIX.png}
\includegraphics{vv.png}
\includegraphics{vr.png}

\newpage

\Large{
  $\sigma=$\ Realized Volatility\ $=\sqrt{\mathbf{W}\mathbf{\Sigma}\mathbf{W^\mathsf{T}}}$; 
  $\overline{\rho}=$\ Realized Average Correlation\ $=\mathbf{W}\mathbf{P}\mathbf{W^\mathsf{T}}$
}
\includegraphics{XLX.png}
\includegraphics{sr.png}
\includegraphics{sc.png}
\includegraphics{ss.png}

\newpage

\Large{
  $\sigma=$\ Realized Volatility\ $=\sqrt{\mathbf{W}\mathbf{\Sigma}\mathbf{W^\mathsf{T}}}$; 
  $\overline{\rho}=$\ Realized Average Correlation\ $=\mathbf{W}\mathbf{P}\mathbf{W^\mathsf{T}}$
}
\includegraphics{sx.png}
\includegraphics{foo.png}
\includegraphics{sdd.png}
\includegraphics{svv.png}

\end{center}
 
\newpage

\underline{\textbf{Notation}}

\begin{itemize}
  \item Minimum: $\wedge$
  \item Maximum: $\vee$
  \item Annualized Return: $\mu$
  \item Annualized Volatility: $\sigma$
  \item Change: $\Delta$
  \item Mean: $$\overline{x}=\frac{1}{N}\sum_{i=1}^Nx_{i}$$
  \item Covariance: $$\sigma_{xy}=\frac{1}{N}\sum_{i=1}^N(x_{i}-\overline{x})(y_{i}-\overline{y})$$
  \item Volatility: $$\sigma_{x}=\sqrt{\sigma_{xx}}$$
  \item Correlation: $$\rho_{xy}=\frac{\sigma_{xy}}{\sigma_{x}\sigma_{y}}$$
  \item Covariance Matrix: $$\mathbf{\Sigma}=
    \begin{bmatrix}
      \sigma_{11} & \ldots & \sigma_{1N} \\
      \vdots & \ddots & \vdots \\
      \sigma_{N1} & \ldots & \sigma_{NN} \\
    \end{bmatrix}
  $$
  \item Correlation Matrix: $$\mathbf{P}=
    \begin{bmatrix}
      \rho_{11} & \ldots & \rho_{1N} \\
      \vdots & \ddots & \vdots \\
      \rho_{N1} & \ldots & \rho_{NN} \\
    \end{bmatrix}
  $$
  \item Portfolio Weights: $$\mathbf{W}=(w_{1} \ldots w_{N})$$
\end{itemize}

\underline{\textbf{Notes}}

\begin{itemize}
  \item \^{}VIX Managed: $Exposure_{t}=2 \wedge \frac{20}{VIX_{t-1}}$
  \item $\mathbf{\Sigma}$ Managed: $Exposure_{t}=2 \wedge \frac{0.15}{a\sqrt{\mathbf{W^*}\mathbf{\Sigma_{t-1}}\mathbf{W^*}^\mathsf{T}}}$ 
    where $\mathbf{W^*}=\underset{\mathbf{W}}{min}\ \overline{\rho_{t-1}}$ s.t. $\sum_{i=1}^N w_{i}=1\ $ and $\ w_{i}>0$
\end{itemize}


\underline{\textbf{Additional $\mathbf{\Sigma}$ Management Techniques:}} \\
\begin{itemize}
  \item Risk Budgeting
  \item Principal Component Analysis
  \item Linear Torsion
  \item Maximum Effective Bets
  \item Maximum Diversification
  \item Minimum Variance 
  \item Minimum Correlation
  \item \ldots
\end{itemize}

\end{document}
